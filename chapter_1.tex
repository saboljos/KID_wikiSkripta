\section{Úvod -- kinetika reaktoru}

\textbf{Kinetika reaktoru} = zkoumá časové chování reaktoru se změnou vstupních parametrů.\\

\textbf{Vstupní parametry} = chápeme primárně $k_{\text{ef}}$, resp. $\rho$, a lze je ovlivnit změnou materiálů či geometrií systému.\\

\textbf{Výstupní parametry} = to, co v systému měříme ($P(t)$, $\Phi (\textbf{r}, t) $ atd.).\\

\textbf{Nulový reaktor} = neboli reaktor nulového výkonu; reaktor pracující v takovém výkonovém rozsahu, že jsou jeho zpětné vazby (ZV) zanedbatelné.

\begin{itemize}
  \item Výzkumné a energetické reaktory sem řadit nelze, jelikož se ZV projevují.
  \item Často složité odlišit, u některých nulových reaktorů lze pozorovat ZV (ve vyšších energetických hladinách) a naopak některé energetické reaktory lze provozovat bez ZV (při minimálním provozním výkonu).
\end{itemize}

\textbf{Zpětná vazba} = proces, díky kterému se změna výstupních parametrů ($P$, $\Phi$) může podílet na změnu vstupních parametrů.\\

\textbf{Dynamika reaktoru} = to samé co kinetika, pouze už uvažuje zapojení ZV.\\

\textbf{Rovnice kinetiky reaktoru} = rovnice popisující závislost změny výstupních parametrů (výsledků) na změně vstupních parametrů.\\

K popisu lze využít transportní rovnici, resp. zjednodušenou difúzní rovnici $\rightarrow$ vede na komplikované soustavy, které nelze v obecném případě řešit analyticky (s projevem heterogenity systému).\\

Řešením jsou \textbf{Rovnice bodové kinetiky}, které zanedbávají změnu prostorového rozložení $\rightarrow$ nastane-li změna na vstupních parametrech (zvětší-li se reaktivita), tak změna výstupních parametrů (např. $\Phi$) se ve všech místech změní stějnou měrou $\rightarrow$ výstupní parametry se tedy pouze škálují a průběh zůstává zachován.\\

\textbf{Rovnice jednobodové kinetiky} = kromě prostorové závislosti se zanedbává i energetické rozdělení $\rightarrow$ vede na 1G rovnice.\\

V reálu to tak není, ale kupodivu dávají rovnice přijatelné výsledky.
